\documentclass[12pt]{article}

\usepackage[margin=1in,footskip=0.25in]{geometry}
\usepackage{amsfonts}
\usepackage{graphicx}
\usepackage{enumitem}
\usepackage{amsmath}

\author{Colin Burdine}
\title{OpenCV Object Tracking Notes}
\date{}


% math macros:
\newcommand{\bv}[1]{\ensuremath\mathbf{#1}}

\begin{document}
\maketitle

\section{Existing Methods}

\subsection{Minimum Output Sum of Squares Error (MOSSE) Filters \cite{bolme_mosse}}

\begin{itemize}
\item \textit{Filter-Based} trackers work by initially selecting a target that is centered within a trackign window in the first frame.

\item The target is then tracked by correlating the filter over a search window in the next frame, and the location that maximizes the correlation is considered the location in the next frame.

\item The principle virtue of the MOSSE Filter is that it is fast, and can perform real-time tracking at high frame rates, making it suitable for edge computing purposes.

\item The correlation is computed using the \textit{Fast Fourier Transform} (FFT; denoted by $\mathcal{F}$). The correlation $\sigma$ is given by the element-wise complex inner product of the filter matrix $h$ and the frame image $f$:

$$\sigma = \mathcal{F}^{-1}(F \odot H^*)\quad \text{ where }\quad F = \mathcal{F}(f),\ H = \mathcal{F}(h)$$

\item The most computationally expensive operation in this process is the 2D FFT and inverse 2D FFT, which can be performed in $\mathcal{O}(n^2\log(n))$ for an $n\times n$ filter.

\item We shall break down the operation of MOSSE filters into some critical sections:

\subsubsection*{Preprocessing}

\item For simplicity, the MOSSE algorithm first reduces the image to a log grayscale. Then, the image is normalized to have pixel intensity values with mean $0$ and std. dev. $1$.

\item Since the 2D FFT projects a rectangular region onto a torus shape, connecting the top with the bottom and the left side with the right side, we want to avoid a filter ``wrapping around" the edges of the image. This is remedied by multiplying the image by a cosine window.

\subsubsection*{Applying the MOSSE Filter}

Ideally, a desired MOSSE filter $H$ satisfies the condition $G = F \odot H^*$, however, we can find the filter $H$ that is closest to this equality in the $L^2$ sense by solving the optimization problem that minimizes the residue:

$$\min_{H^*}\sum_{i}|F_i \odot H_i^* - G_i|^2$$

Since the set of possible filters forms a Hilbert space, the solution (given by the Hilbert projection Theorem) is simply the projection of $G$ onto the basis of $F$, given by:

$$H^* = \frac{\sum_{i} G_i \odot F_i^*}{\sum_{i} F_i \odot F_i^*}$$

\subsubsection*{Regularization}

\item A Mosse filter can fit a single image (e.g. $f_1$) perfectly, which is not desirable in general. Thus, we use the average of the filter trained across all of the images to produce a filter that generalizes better. This averaging is motivated by the idea of \textit{Bootstrap Aggregation}:

$$H^* = \frac{1}{N}\sum_{i} \frac{G_i \odot F_i^*}{F_i \odot F_i^*}$$

\item MOSSE filters can sometimes be unstable for a low number of images; however, we can apply regularization by replacing the denominator $(F_i \odot F_i^*)$ with $(F_i \odot F_i^*) + (\epsilon\bv{1})$, where $\epsilon$ is the regularization parameter and $\bv{1}$ is the matrix of ones. 

\item (Note: Figure 4 in \cite{bolme_mosse} seems to indicate that regularization is not necessary, though it is unclear what the number of images $N$ is.)


\subsubsection*{Initializing and Updating the MOSSE Filter}

\item The MOSSE filters are initialized as described in the previous section, with an initial value of $\epsilon$. According to Figure 4 in \cite{bolme_mosse}, $\epsilon = 0.1$ seems to be a good choice.

\item To update the filter as tracking progresses, a moving average with ``forgetfulness" parameter $\eta \in [0,1]$ is used in both the numerator and denominator. This results in the formula:

$$H_i^* = \frac{A_i}{B_i},\quad \text{ where } \quad A_i = \eta(G_i \odot F_i^*) + (1-\eta)A_{i-1},\ B_i = \eta(F_i \odot F_i^*) + (1-\eta)B_{i-1}$$

\item \cite{bolme_mosse} recommends selecting $\eta = 0.125$.

\subsubsection*{Failure Detection}

\item In order to detect if a tracked object has been lost, we determine a threshold value for the \textit{Peak to Sidelobe Ratio} (PSR). The PSR is computed for a correlation output $G$ as $$PSR(G) = \frac{g_{max} - \mu_{sl}}{\sigma_{sl}}$$ where $g_{max}$ is the point where the correlation is maximized (the object's estimated location), and $(\mu_{sl},\sigma_{sl})$ are the mean and std. dev. of the sidelobe region, which is the region of all pixels excluding those within some window centered about $g_{max}$. Typically, this window size is $11\times 11$ pixels.

\item \cite{bolme_mosse} recommends that when the PSR drops below the threshold of 7.0, the object is lost.

\end{itemize}


\subsection{Generic Object Tracking Using Regression Networks (GOTURN)}

\subsection{MedianFlow Tracking}

\subsection{Discriminative Correlation Filters with Channel and Spatial Reliability (CSRT)}

\subsection{Kernelized Correlation Filters (KCF)}

\bibliographystyle{unsrt}
\bibliography{opencv_tracking}
\end{document}
